\documentclass{ctexart}
\usepackage{newfloat}
\usepackage{longtable, booktabs}
\usepackage{graphicx}

\usepackage{minted}
\usemintedstyle{one-dark}
\setmonofont{Fira Code}
\newmintinline[mono]{python}{breakanywhere,fontsize=\small}
\newminted{python}{breakanywhere,tabsize=2,autogobble,breaklines=true,escapeinside=||,fontsize=\small}
\newenvironment{monos}{\VerbatimEnvironment\begin{pythoncode}}{\end{pythoncode}}

\usepackage{tikz}
\usetikzlibrary{external}
\tikzexternalize
\usepackage{pgfplots}
\pgfplotsset{compat=1.18}

\begin{document}
\section{Multi Label Classification}

如果一个图像可能存在 0 个或更多个类别,这就是一个 multi-label classification 问题。

\subsection{Data}

PASCAL 数据集。

\begin{monos}
from fastai.vision.all import *
path = untar_data(URLs.PASCAL_2007)
df = pd.read_csv(path / 'train.csv')
df.head()
\end{monos}

\begin{longtable}[]{@{}llll@{}}
    \toprule()
      & fname      & labels       & is\_valid \\
    \midrule()
    \endhead
    0 & 000005.jpg & chair        & True      \\
    1 & 000007.jpg & car          & True      \\
    2 & 000009.jpg & horse person & True      \\
    3 & 000012.jpg & car          & False     \\
    4 & 000016.jpg & bicycle      & True      \\
    \bottomrule()
\end{longtable}

\subsubsection{创建 DataBlock}

\begin{monos}
from operator import attrgetter 
# 使用 attrgetter 而非 lambda 避免序列化问题
dblock = DataBlock(get_x = attrgetter('fname'), get_y = attrgetter('labels'))
dsets = dblock.datasets(df)
dsets.train[0] 
# ('005209.jpg', 'car')
\end{monos}

为了真正的打开图片,我们需要使用 BlockType。

\begin{monos}
def get_x(row):
    return path / 'train' / row['fname']
def get_y(row):
    return row['labels'].split(' ')
dblock = DataBlock(blocks = (ImageBlock, MultiCategoryBlock),
                  get_x = get_x,
                  get_y =  get_y)
dsets = dblock.datasets(df)
dsets.train[0]

# (PILImage mode=RGB size=500x375,
# TensorMultiCategory([0., 0., 0., 0., 0., 0., 0., 0., 0., 0., 0., 0., 0., 0., 1., 0., 0., 0.,
#        0., 0.]))
\end{monos}


这里使用了一个 tensor 对象,每个元素 0 表示不是该类别,1 表示是该类别。这称为 one-hot
encoding。这么做的原因在于:我们不能直接使用长度不一致一个类别列表,PyTorch 只能处理
tensor,并且要求长度一致。

从 one-hot encoding 返回类别:

\begin{monos}
idxs = torch.where(dsets.train[0][1] == 1.0)[0]
dsets.train.vocab[idxs]
\end{monos}

最后,使用 \mono!is_valid! 来创建测试集和训练集。以及 \mono!Resize! 和
\mono!aug_transforms! 来进行 data augmentations.

\begin{monos}
def splitter(df):
    """
    返回两个 idxs list, 表示训练和结果集
    """
    train = df.index[~df['is_valid']].to_list()
    valid = df.index[df['is_valid']].to_list()
    return train, valid

dblock = DataBlock(blocks=(ImageBlock, MultiCategoryBlock),
                  splitter=splitter,
                  get_x=get_x,
                  get_y=get_y,
                  item_tfms=Resize(400),
                  batch_tfms=aug_transforms(size=224, min_scale=0.75))
dls = dblock.dataloaders(df)
\end{monos}

\subsection{Learner}

\begin{monos}
    learn = vision_learner(dls, resnet34)
    x, y = dls.train.one_batch()
    activs = learn.model(x)
    activs.shape
\end{monos}

我们创建一个 vision\_learner, 然后将一个 mini-batch 输入模型,得到 (64, 20) 形状的输出。

\begin{itemize}
    \item 64 是 mini-batch 的大小
    \item 20 是所有 category 的可能性
\end{itemize}

通过使用 sigmoid 函数,我们可以将其转换为概率。


\end{document}